\documentclass[aspectratio=169]{beamer}

\usepackage[utf8]{inputenc} 
\usepackage[T1]{fontenc}
\usepackage[brazil]{babel}
\usepackage{amsmath, amssymb, amsthm, mathtools}

% Tema usado
\usetheme{metropolis}

\title{Introdução à Lógica Formal}
\subtitle{CCMP0133 -- Aula 03}
\date{\today}
\author{Prof. Valdigleis S. Costa\\\url{valdigleis.costa@univasf.edu.br}}
\institute{Universidade Federal do Vale do São Francisco\\Colegiado de Ciência da Computação\\\textit{Campus} Salgueiro-PE}

\newtheorem{defi}{Definição}
\newtheorem{teo}{Teorema}
\newtheorem{col}{Corolário}

\begin{document}
	\maketitle
	
	\begin{frame}{Roteiro}
		\tableofcontents
	\end{frame}

	\section{Introdução}
	
	\begin{frame}{Discussão Básica}
		\begin{itemize}
			\item O que é Lógica?\pause
			\item Quais os objetos de interesse no estudo da lógica?\pause
			\begin{itemize}
				\item Argumentos.\pause
				\item Proposições.\pause
				\item Predicados.\pause
			\end{itemize}
		\end{itemize}
	\end{frame}

	\begin{frame}{Os Argumentos}
		\begin{defi}[Argumento]\label{def:Argumento}
			Um argumento é um par formado por dois componentes básicos, a saber:
			\begin{itemize}
				\item[(1)] Um conjunto de frases declarativas, em que cada frase é chamada de premissa.
				\item[(2)] Uma frase declarativa, chamada de conclusão.
			\end{itemize}
		\end{defi}
		\pause
		\begin{alertblock}{Sobre os argumentos:}
			\
			
			Em geral é usado o símbolo $\therefore$ para separar as premissas da conclusão.
		\end{alertblock}
	\end{frame}

	\begin{frame}{Exemplos}
		\begin{itemize}
			\item[a.] 
			\begin{flushleft}
				A sopa foi preparada  sem cebola\\ 
				Toda quarta-feira é servida sopa para as crianças.\\
				Hoje é quinta-feira.\\
				$\therefore$\\
				Ontem as crianças tomaram sopa.
			\end{flushleft}
			\pause
			
			\
			
			\
			\item[b.] 
			\begin{flushleft}
				A lua é feita de queijo\\
				Os ratos comem queijo\\ 
				$\therefore$\\
				O imperador da lua é um rato.
			\end{flushleft}
		\end{itemize}
	\end{frame}

	\begin{frame}{Proposições $\times$ Predicados}
		\begin{defi}[Proposição]\label{def:Proposicao}
			Uma proposição é uma frase declarativa sobre as propriedades de indivíduos específicos em um discurso.
		\end{defi}
		\pause
		\begin{defi}[Predicados]\label{def:Predicados}
			Predicados são frase declarativas sobre as propriedades de indivíduos não específicos em um discurso.
		\end{defi}
	\end{frame}

	\begin{frame}{Exemplos --- Proposições}
		Todas as frases a seguir são proposições:
		\begin{itemize}
			\item[(a)] $3 < 5$.
			\item[(b)] A lua é feita de queijo.
			\item[(c)] Albert Einstein era francês.
			\item[(d)] O Brasil é um país do continente europeu.
		\end{itemize}
		\pause		
		\begin{alertblock}{A frase a seguir é uma proposição: }
			\
			
			Qual é a cor dos olhos de Camila?
		\end{alertblock}
	\end{frame}

	\begin{frame}{Exemplos --- Predicados}
		São exemplos de predicados:
		\begin{itemize}
			\item[(a)] Para qualquer $x \in \mathbb{N}$ tem-se que $x < x + 1$.
			\item[(b)] Para todo $x \in \mathbb{R}$ sempre existem dois números $y_1, y_2 \in \mathbb{R}$ tal que $y_1 < x < y_2$.
			\item[(c)] Existe algum professor cujo nome da mãe é Maria de Fátima.
			\item[(d)] Há um estado brasileiro que não tem litoral.
			\item[(e)] Todo os moradores de Salgueiro são pernambucanos.
		\end{itemize}
	\end{frame}

	\section{Conectivos e Quantificadores}
	\section{Lógica Simbólica}
	\section{Um pouco de semântica}
\end{document}
