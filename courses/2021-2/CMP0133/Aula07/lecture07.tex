\documentclass[aspectratio=169]{beamer}

\usepackage[utf8]{inputenc} 
\usepackage[T1]{fontenc}
\usepackage[brazil]{babel}
\usepackage{amsmath, amssymb, amsthm, mathtools}
\usepackage{multicol}
\usepackage{multirow}

% Tema usado
\usetheme{metropolis}

\title{Métodos para Provas Universais e Existenciais.}
\subtitle{CCMP0133 -- Aula 07}
\date{06 de junho de 2022}
\author{Prof. Valdigleis S. Costa\\\url{valdigleis.costa@univasf.edu.br}}
\institute{Universidade Federal do Vale do São Francisco\\Colegiado de Ciência da Computação\\\textit{Campus} Salgueiro-PE}

\newtheorem{defi}{Definição}
\newtheorem{teo}{Teorema}
\newtheorem{col}{Corolário}
\newtheorem{remark}{Observação}

\begin{document}
	\maketitle
	
	\begin{frame}{Roteiro}
		\tableofcontents
	\end{frame}

	\section{Provas de Generalização}
	
	\begin{frame}{O básico e a formalização}
		\begin{itemize}
			\item O que é uma propriedade generalista ou universal?\pause
			\item Qual a natureza das asserções que são generalizações?\pause
			\begin{itemize}
				\item São predicados!
				\item Possuem o quantificador $\forall$ (para todo) em sua estrutura.
			\end{itemize} 
		\end{itemize}
		\pause
		\begin{defi}[Prova de Generalizações --- PG]
			\
			
			Para provar uma asserção da forma, ``$(\forall x)[P(x)]$'', em que $P(x)$ é uma asserção acerca da variável $x$. Deve-se assumir que a variável $x$ assume como valor um objeto qualquer no universo do discurso de que trata a  generalização, em seguida, provar que a asserção $P(x)$ é verdadeira, usando as propriedades disponível de forma genérica para os objetos do universo do discurso.
		\end{defi}
	\end{frame}

	\begin{frame}{Exemplos}
		Vamos praticar, escolham...
		\begin{itemize}
			\item[(a)] $(\forall x \in \{4n \mid n \in \mathbb{N} \})$[$x$ é par]
			\item[(b)] $(\forall X, Y \subseteq \mathbb{U})$[se $X \neq \emptyset$, então $(X \cup Y) \neq \emptyset$].
			\item[(c)] Demonstre que para todo $n \in \mathbb{Z}$, se $5n$ é ímpar, então $n$ é ímpar.
			\item[(d)] Prove que para todo $A, B, C  \subseteq \mathbb{U}$. Se $A$ e $(B - C)$ são disjuntos, então $(A \cap B) \subseteq C$.
		\end{itemize}
	\end{frame}

	\section{Provas de Existência}
	
	\begin{frame}{O básico e a formalização}
		\begin{itemize}
			\item Qual a natureza das asserções que são existenciais?\pause
			\begin{itemize}
				\item São predicados!
				\item Possuem o quantificador $\exists$ (existe) em sua estrutura.
			\end{itemize} 
		\end{itemize}
		\pause
		\begin{defi}[Prova de existência (PE)]
			\
			
			Para provar uma asserção da forma ``$(\exists x)[P(x)]$'', em que $P(x)$ é uma asserção sobre a variável x. Deve-se exibir um elemento específico ``$a$'' pertencente ao universo do discurso, e mostrar que a asserção $P(x)$ é verdadeira quando $x$ é instanciado como sendo exatamente o elemento $a$, ou seja, deve-se mostrar que $P(a)$ é verdadeira.
		\end{defi}
	\end{frame}

	\begin{frame}{Exemplos}
		Vamos praticar, escolham...
		\begin{itemize}
			\item[(a)] $(\exists n \in \mathbb{N})$[O dobro do quadrado de $x$ é exatamente 32]
			\item[(b)] $(\exists  x, y \in \mathbb{Z})$[$x > 0$ e $y + x = -2x$].
			\item[(c)] $(\exists X \in \wp(A))[(\forall Y \in \wp(A))[X \cup Y = Y]]$.
		\end{itemize}
	\end{frame}
	
	\section{Provas de Unicidade}
	
	\begin{frame}{O básico e a formalização}
		\begin{itemize}
			\item O que seria uma prova de unicidade?\pause
			\begin{itemize}
				\item Consiste de uma prova existencial e que um \textbf{único} elemento do discurso satisfaz a propriedade ``estudada''.
			\end{itemize} 
		\end{itemize}
		\pause
		\begin{defi}[Prova de unicidade (PU)]\label{def:ProvaUnicidade}
			
			Uma prova de unicidade consiste em provar uma asserção da forma ``$(\exists x)[P(x) \land (\forall y)[P(y) \Rightarrow x = y]]$'', em que $P$ é uma asserção sobre os elementos do discurso. Para tal primeiro deve-se demonstrar que a asserção ``$(\exists x)[P(x)]$'' é verdadeira, e depois prova que a generalização $(\forall y)[P(y) \Rightarrow x = y]$ também é verdadeira.
		\end{defi}
	\end{frame}

	\begin{frame}{Exemplos}
		Vamos praticar, escolham...
		\begin{itemize}
			\item[(a)] $(\exists! A \in \wp(U))[(\forall B \in \wp(U))[A \cap B = B]]$.
			\item[(b)] $(\exists! n \in \mathbb{N})[(\forall x)[n + x = 0]]$.
			\item[(c)] $(\forall x \in \mathbb{Z})[(\exists! y \in \mathbb{Z})[x + y = 0]]$.
		\end{itemize}
	\end{frame}
	
\end{document}
