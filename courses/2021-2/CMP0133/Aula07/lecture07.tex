\documentclass[aspectratio=169]{beamer}

\usepackage[utf8]{inputenc} 
\usepackage[T1]{fontenc}
\usepackage[brazil]{babel}
\usepackage{amsmath, amssymb, amsthm, mathtools}
\usepackage{multicol}
\usepackage{multirow}

% Tema usado
\usetheme{metropolis}

\title{Métodos para Provas Universais e Existenciais.}
\subtitle{CCMP0133 -- Aula 07}
\date{06 de junho de 2022}
\author{Prof. Valdigleis S. Costa\\\url{valdigleis.costa@univasf.edu.br}}
\institute{Universidade Federal do Vale do São Francisco\\Colegiado de Ciência da Computação\\\textit{Campus} Salgueiro-PE}

\newtheorem{defi}{Definição}
\newtheorem{teo}{Teorema}
\newtheorem{col}{Corolário}
\newtheorem{remark}{Observação}

\begin{document}
	\maketitle
	
	\begin{frame}{Roteiro}
		\tableofcontents
	\end{frame}

	\section{Provas de Generalização}
	
	\begin{frame}{O básico e a formalização}
		\begin{itemize}
			\item O que é uma propriedade generalista ou universal?\pause
			\item Qual a natureza das asserções que são generalizações?\pause
			\begin{itemize}
				\item São predicados!
				\item Possuem o quantificador $\forall$ (para todo) em sua estrutura.
			\end{itemize} 
		\end{itemize}
		\pause
		\begin{defi}[Prova de Generalizações --- PG]
			\
			
			Para provar uma asserção da forma, ``$(\forall x)[P(x)]$'', em que $P(x)$ é uma asserção acerca da variável $x$. Deve-se assumir que a variável $x$ assume como valor um objeto qualquer no universo do discurso de que trata a  generalização, em seguida, provar que a asserção $P(x)$ é verdadeira, usando as propriedades disponível de forma genérica para os objetos do universo do discurso.
		\end{defi}
	\end{frame}

	\begin{frame}{Exemplos}
		Vamos praticar, escolham...
		\begin{itemize}
			\item[(a)] $(\forall x \in \{4n \mid n \in \mathbb{N} \})$[$x$ é par]
			\item[(b)] $(\forall X, Y \subseteq \mathbb{U})$[se $X \neq \emptyset$, então $(X \cup Y) \neq \emptyset$].
			\item[(c)] Demonstre que para todo $n \in \mathbb{Z}$, se $5n$ é ímpar, então $n$ é ímpar.
			\item[(d)] Prove que para todo $A, B, C  \subseteq \mathbb{U}$. Se $A$ e $(B - C)$ são disjuntos, então $(A \cap B) \subseteq C$.
		\end{itemize}
	\end{frame}


	\section{Provas de Existência}
	\section{Provas de Unicidade}
	
\end{document}
