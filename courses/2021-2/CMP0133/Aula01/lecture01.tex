\documentclass{beamer}

\usepackage[utf8]{inputenc} 
\usepackage[T1]{fontenc}
\usepackage[brazil]{babel}
\usepackage{amsmath, amssymb, amsthm, mathtools}



% Tema usado
\usetheme{metropolis}

\title{Apresentação do Curso}
\subtitle{CCMP0133 -- Aula 01}
\date{16 de Maio de 2022}
\author{Prof. Valdigleis S. Costa\\\url{valdigleis.costa@univasf.edu.br}}
\institute{Universidade Federal do Vale do São Francisco\\Colegiado de Ciência da Computação\\\textit{Campus} Salgueiro-PE}

 \begin{document}
	\maketitle
	\begin{frame}{Roteiro}
		\tableofcontents
	\end{frame}
	
	\section{Sobre Linguagens}

	\begin{frame}{Questionamentos Básicos}
		\begin{itemize}
			\item O que é uma linguagem?\pause { \color{red} Segundo o dicionário Aurélio}:
			\begin{itemize}
				\item Faculdade dos indivíduos de se comunicar entre si, exprimindo pensamentos e sentimentos por palavras, que podem ser escritas, quando necessário.
				\item Sistema de símbolos que permite a representação de uma informação\pause
			\end{itemize}
			\item O que seriam as linguagens formais?\pause
			\begin{itemize}
				\item As linguagens em que todas as sentenças expressas em tal linguagens de um forma precisa, impassível de ambiguidade de escrita\footnote{No sentido de que é sempre possível saber quando uma sentença está bem escrita do ponto de visto léxico.} ou de interpretação dúbia.
			\end{itemize}
			\item O que são as linguagens naturais?\pause
			\begin{itemize}
				\item São as linguagens que não são formais.
			\end{itemize}
		\end{itemize}
	\end{frame}

	\begin{frame}{Os Componentes Básicos das Linguagem Formais}
		\begin{itemize}
			\item Alfabeto: {\color{red}Aglomerado} de símbolos usados para escrever as palavras da linguagem.\pause
			\item Gramática: {\color{red}Aglomerado} de regras que define como escrever as palavras da linguagem.\pause
		\end{itemize}
	\end{frame}

	\section{Referências}
	
	\begin{frame}[allowframebreaks]{Referências}
		\bibliography{references}
	\end{frame}

\end{document}
