\documentclass{beamer}

\usepackage[utf8]{inputenc} 
\usepackage[T1]{fontenc}
\usepackage[portuguese]{babel}
\usepackage{amsmath, amssymb, amsthm, mathtools}


% Tema usado
\usetheme{metropolis}

\title{Apresentação do Curso}
\subtitle{CCMP0133 -- Apresentação}
\date{16 de Maio de 2022}
\author{Prof. Valdigleis S. Costa}
\institute{Universidade Federal do Vale do São Francisco\\Colegiado de Ciência da Computação\\\textit{Campus} Salgueiro-PE}

 \begin{document}
	\maketitle
	\begin{frame}{Roteiro}
		\tableofcontents
	\end{frame}
	
	\section{O que é este curso?}

	\begin{frame}{Informações básicas}
		\begin{itemize}
			\item \textbf{Código}: CCMP0133.\pause
			\item \textbf{Nome}: Matemática Discreta.\pause
			\item \textbf{Tipo}: Teórico.\pause
			\item \textbf{Requisitos}: Não possui.\pause
			\item \textbf{É necessário para}: Lógica Clássica para Computação, Análise de Algoritmos, Computabilidade e Decidibilidade e outras.\pause
			\item \textbf{Avaliações}\footnote{Todas individuais.}:
			\begin{itemize}
				\item[$p_1$] realizada no dia {\color{blue}26/06/2022}.
				\item[$p_2$] realizada no dia {\color{blue}25/07/2022}.
				\item[$p_3$] realizada no dia {\color{blue}29/08/2022}.
			\end{itemize}
		\end{itemize}
	\end{frame}

	\begin{frame}{Objetivos e Metodologia}
		\begin{block}{Os objetivos:}\pause
			\begin{itemize}
				\item Apresentar ao aluno os principais tópicos da matemática que são necessários para a Ciência da Computação.
				\item Tornar o aluno proficiente no pensar e no escrever de provas matemáticas.
				\item Apresentar as diferenças básicas entre os aspectos sintáticos e semânticos da matemática e das linguagens de programação.
			\end{itemize}
		\end{block}
		\pause
		\begin{block}{A metodologia:}\pause
			\begin{itemize}
				\item Aulas expositivas {\color{red}totalmente} presenciais.
				\item Listas de exercícios.
				\item Seção (ou horário de atendimento) para dúvidas.
			\end{itemize}
		\end{block}
	\end{frame}

	\begin{frame}{Sobre a seção de dúvidas}
		\begin{block}{Atendimento Presencial:}
			\begin{itemize}
				\item \textbf{Com o professor}: quinta-feira (08:00--10:00 am), em local definido posteriormente.
				\item \textbf{Com o monitor}: a definir\footnote{Isto depende da possível aprovação de projeto de monitoria e de um monitor.}.
			\end{itemize}
		\end{block}
		\pause
		\begin{block}{Comunicação eletrônica:}
			\begin{itemize}
				\item \textbf{Com o professor}: \url{valdigleis.costa@univasf.edu.br}
				\item \textbf{Com o monitor}: a definir.
			\end{itemize}
		\end{block}
	\end{frame}

	\section{Os conteúdos}
	
	\begin{frame}{Conteúdo}
		\begin{block}{Ementa:}
			Introdução à Lógica Formal,  Métodos de demonstração, Teoria ingênua dos conjuntos, Relações e funções, Ordem e equivalência, Cardinalidade, Indução e Estruturas Indutivamente Geradas, Recursividade e Relações de Recorrência, Estruturas algébricas, Reticulados e Álgebras booleanas.
		\end{block}
	\end{frame}

	\begin{frame}{Conteúdo Programático}
		\begin{eqnarray*}
			{\color{blue}1^a \text{ prova}} & & \left\{\begin{array}{ll}
				\text{Teoria Ingênua dos Conjuntos}&\\	
				\text{Introdução à Lógica Formal}&\\
				\text{Métodos de Demonstração}&\\
				\text{Relações}&
			\end{array}\right.\\\pause
			{\color{green}2^a \text{ prova}} & & \left\{\begin{array}{ll}
				\text{Funções}&\\	
				\text{Recursão e Indução}&\\
				\text{Relações de Ordem e Equivalência}&\\
				\text{Cardinalidade}&
			\end{array}\right.\\\pause
			{\color{red}3^a \text{ prova}} & & \left\{\begin{array}{ll}
			\Sigma\text{-Álgebras}&\\	
			\text{Estruturas Algébricas Básicas}&\\
			\text{Morfismos Algébricos}&\\
			\text{Introdução à Teoria de Reticulados}&\\
			\text{Álgebras Booleanas}&
			\end{array}\right.
		\end{eqnarray*}
	\end{frame}
	
	\section{O Referencial Literário}
	
	\begin{frame}{Bibliografia}
		\begin{block}{Principal}
			\begin{itemize}
				\item GERSTING, Judith L. Fundamentos matemáticos para a ciência da computação. LTC, 2001.
				\item HALMOS, Paul R. Teoria ingênua dos conjuntos. Editora Ciência Moderna, 2001.
				\item PAPAVERO, Nelson; ABE, Jair Minoro. Teoria intuitiva dos conjuntos. Makron Books $\&$ McGrawHill. São Paulo. 1991.
				\item LIPSCHUTZ, Seymour; DA SILVA, Fernando Vilain Heusi. Teoria dos conjuntos. 1967.
			\end{itemize}
		\end{block}
	\end{frame}

	\begin{frame}{Bibliografia}
		\begin{block}{Complementar}
			\begin{itemize}
				\item CARMO, José; GOUVEIA, Paula; Dionísio, Francisco Miguel; Elementos de Matemática Discreta, 1ª edição, 2012.
				\item MAKINSON, D.; Sets, Logic and Maths for Computing, 2nd edition, Springer, 2012.
				\item EPP, S. S.; Discrete Mathematics with Applications, 3rd edition, Brooks Cole, 2003
				\item {\color{red}Costa, Valdigleis S, Computação Formal -- Um compêndio dos Fundamentos Matemáticos da Computação}, disponível em \url{https://profvaldi.site/manuscrito}
			\end{itemize}
		\end{block}
	\end{frame}

\end{document}
