\documentclass[aspectratio=169]{beamer}

\usepackage[utf8]{inputenc} 
\usepackage[T1]{fontenc}
\usepackage[brazil]{babel}
\usepackage{amsmath, amssymb, amsthm, mathtools}
\usepackage{multicol}
\usepackage{multirow}

% Tema usado
\usetheme{metropolis}

\title{Método Reductio ad absurdum (Redução ao Absurdo)}
\subtitle{CCMP0133 -- Aula 06}
\date{30 de maio de 2022}
\author{Prof. Valdigleis S. Costa\\\url{valdigleis.costa@univasf.edu.br}}
\institute{Universidade Federal do Vale do São Francisco\\Colegiado de Ciência da Computação\\\textit{Campus} Salgueiro-PE}

\newtheorem{defi}{Definição}
\newtheorem{teo}{Teorema}
\newtheorem{col}{Corolário}
\newtheorem{remark}{Observação}

\begin{document}
	\maketitle
	
	\begin{frame}{Roteiro}
		\tableofcontents
	\end{frame}

	\section{Introdução}
	
	\begin{frame}{Perguntas Básicas}
		\begin{itemize}
			\item  O que é um Absurdo?\pause
			\item  O que seria reduzir ao absurdo?\pause
			\item  Como funcionaria uma prova que reduz ao absurdo? 
		\end{itemize}
	\end{frame}

	\section{A Definição}
	
	\begin{frame}{Formalizando}
		\begin{defi}[Prova por Redução ao Absurdo --- RAA]
			\
			
			A metodologia para uma demonstração por redução ao absurdo de uma asserção $\alpha$, consiste em supor que não $\alpha$ é uma hipótese verdadeira, então deduzir um absurdo (ou contradição). Em seguida concluir que dado que a partir de não $\alpha$ foi produzido um absurdo pode-se afirma que $\alpha$ é verdadeiro.
		\end{defi}
	\end{frame}

	\section{Seção Prática}

	\begin{frame}{Exemplos}
		Escolham...
		\begin{itemize}
			\item Não existe um inteiro $n$ que seja o maior inteiro.
			\item Se $3n + 2$ é impar, então $n$ é impar.
			\item Se $n^2$ é par, então $n$ é par.
			\item Não existe um $n$ que seja par e impar ao mesmo tempo.
			\item Dado $a,b \in \mathbb{Z}$. Se $a$ divide $b$, então $ab$ divide $b^2$.
			\item Dado $n = ab$ sendo $a, b \in \mathbb{Z}_+^*$. Prove que $a \leq \sqrt{n}$ ou $b \leq \sqrt{n}$.
			\item Não existe um programa $P$ jogador de xadrez que sempre vença.
		\end{itemize}
	\end{frame}

\end{document}
